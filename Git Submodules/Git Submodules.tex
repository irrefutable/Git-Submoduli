\documentclass {article}
\usepackage[utf8]{inputenc}
\begin{document}

\title{Git Submodules}
\author{Antonio Cvitković, Marko Juračić - Tehnički Fakultet, Rijeka}
\date{Veljača 2019}

\maketitle

\newpage

Sadržaj:


\begin{enumerate}
    \item Uvod ........................................................................ 3
    \item Pregled Literature .................................................... 4
        \begin{enumerate}
        \item Korištenje Submodula ...................................... 4
        \item Kloniranje projekta sa submodulima ................ 6
        \item Rad sa projektima uz submodule .......................7
        \end{enumerate}
    \item Metodologija................................................................11
    \item Rasprava.....................................................................11
    \item Zaključak.....................................................................11
    \item Popis Literature..........................................................12
\end{enumerate}

\newpage

\begin{enumerate}
    \item \textbf{Uvod}
\end{enumerate}


Kada radimo neki projekt, često nam se događa da bismo htjeli prvo napraviti jedan manji projekt i zatim ga integrirati u postojeći. Možda je to knjižnica koju vaš kolega stvara dok vi radite na ostatku projekta. Ali tu se stvara problem rada na dva projekta, nezavisni jedan od drugog.
\newline
Npr. Zamislite da radite određenu web stranicu na kojoj korisnik unosi određeni broj koji prolazi kroz određene algoritme kako bi dobili traženi odgovor i umjesto da napravimo određeni kod, možemo se koristiti knjižnicama iz drugog projekta. No ta knjižnica neće uvijek radit ono što želite sama po sebi, pa tako tim drugim projektom zapravo proširujemo značenje knjižnice.
\newline
Git ovako riješava  taj problem korištenjem sub modula. Submoduli vam dopuštaju da određeni repozitorij spremate kao pod repozitorij nekog većeg repozitorija.
\newline
Ovo vam dopušta kloniranje drugog projekta kao pod projekt koji možete integrirati u vaš projekt, dok vam svi commitovi razlicitih projekata ostanu razdvojeni, zasebni.
\newline
Ovo se najčešće koristi kod velikih timova ljudi koji rade na određenim projektima bilo to poslovnim ili rekreacijskim kada u timu postoji nekolicina ili vise od desetak prinosnika, suradnika (contributors).
\newline
Ovim seminarom objasnit ćemo korištenje git submodula unutar projekata.

\newpage

\begin{enumerate}
    \item \textbf{Pregled Literature}
        \begin{enumerate}
            \item \textbf{Korištenje Submodula}
  

Kako bismo prošli kroz proces korištenja submodula koristit ćemo projekt kojeg smo ranije razdvojili u glavi dio i više manjih pod dijela.
\newline
To možemo napraviti dodavanjem postojećeg repozitorija kao submodul repozitorija na kojem trenutno radimo.
\newline
Dodavanje novog submodula možemo izvršiti naredbom \textbf{\emph{git submodule add}} gdje na kraju naredbe dodajemo apsolutni ili relativni URL projekta kojeg želimo pratiti pri stvaranju ovog projekta. Za ovaj primjer dodat ćemo knjižnicu DbConnector iz jednog online repozitorija.

\begin{quote}
   \$ git submodule add \url{https://github.com/chaconinc/DbConnector}
    \newline Cloning into 'DbConnector'...
    \newline remote: Counting objects: 11, done.
    \newline remote: Compressing objects: 100\% (10/10), done.
    \newline remote: Total 11 (delta 0), reused 11 (delta 0)
    \newline Unpacking objects: 100\% (11/11), done.
    \newline Checking connectivity... done.

\end{quote}

Prema zadanoj funkciji naredbe, submodul će dodat pod projekt u direktorij pod nazivom istim kao i repozitorij kojeg koristimo za rad - u ovom slučaju "DbConnector". Usprkos tome, možemo na kraj naredbe git submodule add i specifikacije web stranice koju želimo postaviti kao sub modul, uvijek možemo dodati putanju kojom želimo da se submodul spremi.
\newline
\newline
Sada ako pokrenemo \textbf{\emph{git status}} naredbu primjetit ćemo par stvari:

\begin{quote}
\$ git status
On branch master
Your branch is up-to-date with 'origin/master'.

Changes to be committed:
  \newline(use "git reset HEAD file..." to unstage)

	new file:   .gitmodules\newline
	new file:   DbConnector
\end{quote}

\newline

Prvo, primjetili smo novu .gitmodules datoteku.
\newline To je datoteka koja spaja web stranicu na kojoj se nalazi projekt (u ovom slučaju knjižnica) kojeg koristimo i putanju do direktorija gdje smo spremili submodul kako bi jedan sa drugim mogli komunicirati.
\newline
Bitno je napomenuti da ako koristimo više različitih submodula da će svi biti navedeni pri izvršavanju .git status naredbe.
\newline
Također, submoduli se provlače kroz različite verzije projekta tako da su uvijek prisutni. Ovo je izrazito korisno drugim ljudima koji kloniraju projekt zato što znaju gdje mogu pronaći određeni projekt korišten u sub modulu.

\newpage

Osim .gitmodules datoteke možemo primjetiti i datoteku DbConnector. Ova datoteka se sastoji od svega što smo dodali u naš sub modul. Ako iskoristimo \textbf{\emph{git diff}} na toj datoteci primjetit ćemo nešto zanimljivo.

\begin{quote}
    \ git diff --cached DbConnector
    \newline diff --git a/DbConnector b/DbConnector
    \newline new file mode 160000
   \newline index 0000000..c3f01dc
    \newline--- /dev/null
   \newline +++ b/DbConnector
   \newline @@ -0,0 +1 @@
   \newline +Subproject commit c3f01dc8862123d317dd46284b05b6892c7b29bc
\end{quote}

Iako je DbConnector pod direktorij u direktoriju s kojim radimo, git ga učitava kao submodul i ne prati njegov sadržaj osim ako se ne nalazimo baš unutar samog direktroija. Uz to, git ga vidi kao commit na repozitoriju s kojim radimo.
\newline Ako želimo dobiti malo lijepši ispis, na kraju naredbe možemo dodati "--submodule" opciju na git diff.

\begin{quote}
   \$ git diff --cached --submodule
   \newline diff --git a/.gitmodules b/.gitmodules
  \newline  new file mode 100644
  \newline  index 0000000..71fc376
  \newline  --- /dev/null
  \newline  +++ b/.gitmodules
   \newline @@ -0,0 +1,3 @@
   \newline +[submodule "DbConnector"]
  \newline  +       path = DbConnector
  \newline  +       url = https://github.com/chaconinc/DbConnector
  \newline  Submodule DbConnector 0000000...c3f01dc (new submodule)
\end{quote}


Sada kada razumijemo koja je funkcija datoteka unutar submodula, možemo napraviti prvi commit.

\begin{quote}
    \$ git commit -am 'added DbConnector module'
    \newline [master fb9093c] added DbConnector module
    \newline 2 files changed, 4 insertions(+)
    \newline create mode 100644 .gitmodules
    \newline create mode 160000 DbConnector
\end{quote}
Primjetiti ćemo \emph{mode 160000} kod datoteke DbConnector. To je stanje koje označava da commit zapisujemo kao promjenu na direktoriju, a ne na pod direktoriju ili datoteci.

\newpage

    \item {\textbf{Kloniranje projekta sa submodulima}}
    
Sada ćemo klonirat projekt koji već u sebi ima submodule. Prema zadanome, dobivamo i direktorije koji sadrže submodule, ali u njima ne dobivamo nikakve datoteke.

    \begin{quote}
        \$ git clone https://github.com/chaconinc/MainProject
    \newline Cloning into 'MainProject'...
    \newline remote: Counting objects: 14, done.
    \newline remote: Compressing objects: 100% (13/13), done.
    \newline remote: Total 14 (delta 1), reused 13 (delta 0)
    \newline Unpacking objects: 100% (14/14), done.
    \newline Checking connectivity... done.
    \newline \$ cd MainProject
    \newline \$ ls -la
    \newline total 16
    \newline drwxr-xr-x   9 schacon  staff  306 Sep 17 15:21 .
    \newline drwxr-xr-x   7 schacon  staff  238 Sep 17 15:21 ..
    \newline drwxr-xr-x  13 schacon  staff  442 Sep 17 15:21 .git
    \newline -rw-r--r--   1 schacon  staff   92 Sep 17 15:21 .gitmodules
    \newline drwxr-xr-x   2 schacon  staff   68 Sep 17 15:21 DbConnector
    \newline -rw-r--r--   1 schacon  staff  756 Sep 17 15:21 Makefile
    \newline drwxr-xr-x   3 schacon  staff  102 Sep 17 15:21 includes
    \newline drwxr-xr-x   4 schacon  staff  136 Sep 17 15:21 scripts
    \newline drwxr-xr-x   4 schacon  staff  136 Sep 17 15:21 src
    \newline \$ cd DbConnector/
    \newline \$ ls
    \newline \$
    \end{quote}

DbConnector direktorij se tu nalazi, ali je prazan. Kako bi ga ispunili zadanim submodulima moramo ih prvo inicijalizirati naredbom \textbf{\emph{git submodule init}}, a zatim naredbom \textbf{\emph{git submodule update}} donosimo sve datoteke koje se nalaze u posljednjoj verziji repozitorija kojeg kloniramo i sada možemo pregledati sve commitove kojeg smo napravili u našem velikom, tzv. "super projektu".

    \begin{quote}
        \$ git submodule init
\newline Submodule 'DbConnector' (https://github.com/chaconinc/DbConnector) registered for path 'DbConnector'
\newline \$ git submodule update
\newline Cloning into 'DbConnector'...
\newline remote: Counting objects: 11, done.
\newline remote: Compressing objects: 100\% (10/10), done.
\newline remote: Total 11 (delta 0), reused 11 (delta 0)
\newline Unpacking objects: 100\% (11/11), done.
\newline Checking connectivity... done.
\newline Submodule path 'DbConnector': checked out 'c3f01dc8862123d317dd4...'
    \end{quote}
    
Sada se DbConnector je u istom stanju kao i kada smo commitali ranije.

\newpage
Postoji još jedan način da ovo učinimo koji je malo lakši, a zahtjeva da dodamo "--recurse--submodules" na kraju git clone naredbe i time automatski inicijaliziramo postojeće submodule repozitorija kojeg kloniramo.
\newline
    \begin{quote}
        \$ git clone --recurse-submodules https://github.com/chaconinc/MainProject
\newline Cloning into 'MainProject'...
\newline remote: Counting objects: 14, done.
\newline remote: Compressing objects: 100\% (13/13), done.
\newline remote: Total 14 (delta 1), reused 13 (delta 0)
\newline Unpacking objects: 100\% (14/14), done.
\newline Checking connectivity... done.
\newline Submodule 'DbConnector' (https://github.com/chaconinc/DbConnector) registered for path 'DbConnector'
\newline Cloning into 'DbConnector'...
\newline remote: Counting objects: 11, done.
\newline remote: Compressing objects: 100\% (10/10), done.
\newline remote: Total 11 (delta 0), reused 11 (delta 0)
\newline Unpacking objects: 100\% (11/11), done.
\newline Checking connectivity... done.
\newline Submodule path 'DbConnector': checked out 'c3f01dc8862123d317dd4...'
\newline
    \end{quote}
    
\item {\textbf{Rad sa projektima uz submodule}}
Sada kada smo klonirali projekt sa submodulima, možemo zajedno sa svojim suradnicima raditi na projektu sa submodulima i na glavnom projektu.
\newline Najjednostavniji model korištenja submodula je iskorištavanje pod projekta iz kojeg želimo dobivati ažuriranja s vremena na vrijeme, ali u isto vrijeme ne mijenjamo ništa pri checkoutu.\newline To ćemo objasniti kroz primjer.
\newline Ako želimo pregledati promjene na submodulu, možemo otići do tog direktorija i iskoristiti \emph{git fetch} i \emph{git merge} naredbe kako bismo uzvodno spojili grane da ažuriraju lokalni kod.

    \begin{quote}
        \$ git fetch
\newline From https://github.com/chaconinc/DbConnector
   \newline c3f01dc..d0354fc  master     -> origin/master
\newline \$ git merge origin/master
\newline Updating c3f01dc..d0354fc
\newline Fast-forward
 \newline scripts/connect.sh | 1 +
 \newline src/db.c           | 1 +
 \newline 2 files changed, 2 insertions(+)
    \end{quote}
    
    \newpage
Sada ako odemo natrag na glavni projekt možemo pokrenuti "git diff --submodule" naredbu i vidjeti koji su submoduli bili ažurirani i dobijemo listu commitova koji su dodani. Ako ne želimo koristiti nastavak "--submodule" svaki puta kada pokrećemo "git diff" naredbu možemo taj nastavak postaviti kao zadani mijenjajući vrijednost diff.submodule konfiguracije upisom vrijednosti"log".

    \begin{quote}
        \$ git config --global diff.submodule log
\newline\$ git diff
\newline Submodule DbConnector c3f01dc..d0354fc:
\newline  - more efficient db routine
 \newline - better connection routine
    \end{quote}
    
Ako sada napravimo commit, spremit ćemo promijene novog koda na submodul i svaki put kada ga netko ažurira ponovno će se spremit promijene.
\newline Postoji lakši način da to ostvarimo ako ne preferiramo manualno dohvatati i sjediniti podatke u pod direktorij. Ako iskoristimo \textbf{\emph{git submodule update --remote}} git će pretražiti submodul i dohvatiti nam traženo ažuriranje.
\newline 
\begin{quote}
    \$ git submodule update --remote DbConnector
\newline remote: Counting objects: 4, done.
\newline remote: Compressing objects: 100\% (2/2), done.
\newline remote: Total 4 (delta 2), reused 4 (delta 2)
\newline Unpacking objects: 100\% (4/4), done.
\newline From https://github.com/chaconinc/DbConnector
\newline         3f19983..d0354fc  master     - origin/master
\newline Submodule path 'DbConnector': checked out 'd0354fc054692d39...'
\newline
\end{quote}
Ova naredba pretpostavlja prema zadanome da želite ažuriranje izvršiti na glavnoj grani submodule repozitorija. Ovo možemo promijeniti, primjerice, postavljajući tako da DbConnector submodul prati "stabilnu" granu repozitorija u .gitmodules datoteci kako bi svi mogli vidjeti promjene koje se događaju na njoj ili samo lokalno na .git/config datoteku.
\newline
    \begin{quote}
        \$ git config -f .gitmodules submodule.DbConnector.branch stable

\$ git submodule update --remote
\newline remote: Counting objects: 4, done.
\newline remote: Compressing objects: 100\% (2/2), done.
\newline remote: Total 4 (delta 2), reused 4 (delta 2)
\newline Unpacking objects: 100\% (4/4), done.
\newline From https://github.com/chaconinc/DbConnector
\newline 27cf5d3..c87d55d  stable -> origin/stable
\newline Submodule path 'DbConnector': checked out 'c87d55d4c6d4b05ee3...'
    \end{quote}
\newpage

Ako iskoristimo parametre "-f .gitmodules" promjene će samo biti vidljive na lokalnoj grani, no inače, želimo da svi suradnici imaju pristup pregledu promjena.
\newline Sada ponovno možemo pokrenuti git status naredbu i git će nam prikazati nove promjene (commitove) na submodulu.
\newline 
    \begin{quote}
        \$ git status
\newline On branch master
\newline Your branch is up-to-date with 'origin/master'.
\newline 
\newline Changes not staged for commit:
\newline   (use "git add <file>..." to update what will be committed)
\newline   (use "git checkout -- <file>..." to discard changes in working directory)
\newline 
 \newline  modified:   .gitmodules
\newline   modified:   DbConnector (new commits)
\newline  
\newline no changes added to commit (use "git add" and/or "git commit -a")
\newline
    \end{quote}
Ako postavimo konfiguracijske postavke na \emph{status.submodulesummary} git će nam također pokazati kratki sažetak svih promjena na submodulima.
\newline
    \begin{quote}
        \$ git config status.submodulesummary 1

\$ git status
\newline branch master
\newline Your branch is up-to-date with 'origin/master'.
\newline
\newline Changes not staged for commit:
 \newline (use "git add <file>..." to update what will be committed)
\newline  (use "git checkout -- <file>..." to discard changes in working directory)
\newline
\newline	modified:   .gitmodules
\newline	modified:   DbConnector (new commits)
\newline
\newline Submodules changed but not updated:
\newline
\newline* DbConnector c3f01dc...c87d55d (4):
  \newline- catch non-null terminated lines
  \newline
    \end{quote}
   
\newpage 
Sada ako upišemo "git diff" naredbu primjetit ćemo da smo promijenili našu .gitmodules datoteku i da postoji nekolicina commitova koje smo povukli i spremni su za daljnje commitanje na našem submodule projektu.
\newline 
    \begin{quote}
        \$ git diff
\newline diff --git a/.gitmodules b/.gitmodules
\newline index 6fc0b3d..fd1cc29 100644
\newline --- a/.gitmodules
\newline +++ b/.gitmodules
\newline @@ -1,3 +1,4 @@
\newline  [submodule "DbConnector"]
\newline         path = DbConnector
\newline         url = https://github.com/chaconinc/DbConnector
\newline +       branch = stable
\newline  Submodule DbConnector c3f01dc..c87d55d:
\newline   - catch non-null terminated lines
\newline   - more robust error handling
\newline   - more efficient db routine
\newline   - better connection routine
\newline 
    \end{quote}
Ovo je vrlo korisno s obzirom da možemo vidjeti cijeli log svih commitova koje ćemo upravo commitat u našem submodulu. Čak i nakon samog commitanja možemo pročitati informacije loga kada pokrenemo \textbf{\emph{git log -p}}.
\newline 
    \begin{quote}
        \$ git log -p --submodule
\newline commit 0a24cfc121a8a3c118e0105ae4ae4c00281cf7ae
\newline Author: Marko Juracic markojuracic@gmail.com
\newline Date:   Wed Sep 17 16:37:02 2014 +0200
\newline     updating DbConnector for bug fixes
\newline diff --git a/.gitmodules b/.gitmodules
\newline index 6fc0b3d..fd1cc29 100644
\newline --- a/.gitmodules
\newline +++ b/.gitmodules
\newline @@ -1,3 +1,4 @@
\newline  [submodule "DbConnector"]
\newline         path = DbConnector
\newline         url = https://github.com/chaconinc/DbConnector
\newline +       branch = stable
\newline Submodule DbConnector c3f01dc..c87d55d:
\newline   - catch non-null terminated lines
\newline   - more robust error handling
\newline   - more efficient db routine
\newline   - better connection routine
\newpage
    \end{quote}
      \end{enumerate}
\item {\textbf{Metodologija}}
\newline
\newline Pri izradu seminara, oba suradnika od prije su imali određenu razinu znanja kolegija vezanu za korištenje latexa i gita. Pronađena literatura iskorištena je na način da su studenti izvršili naredbe kroz kontrolnu konzolu i potvrdili tvrdnje prijašnjih istraživanja i uz samo istraživanje, naučili nekolicinu novih stvari.
\newline
\item {\textbf{Rasprava}}
\newline
\newline Korištenje git submodula na prvi pogled ima svoje dobre strane i mane. \newline Prvim korištenjem i razumijevanjem git submodula korisnik često može završiti izgubljen u informacijama koje su pred njega bačene, no uz strpljenje i razmišljanje koncept git submodula se vrlo lako može pohvatati i više je nego koristan za rad u velikim grupama suradnika pri izradnji bilo kakvog projekta. Samom činjenicom da smo u mogućnosti raditi različite promjene u isto vrijeme na lokalnoj bazi radeđi sa istim datotekama pri korištenju git submodula je izuzetno dobro. Osim toga, git submoduli nam dopuštaju lakši pristup udaljenim repozitorijima i time nam znatno olakšavaju proces izgradnje super projekata.
\newline
\item {\textbf{Zaključak}} \newline
\newline
Git Submoduli su izrazito moćan alat koji dolzvoljava pregledan ispis svih promjene na različitim lokalnim granama submodulima. Uz mogućnost prelaženja iz verzije u verziju bez korištenja ikakvih dodatnih naredbi daljnje doprinosti očuvanju povijesti svih promijena. Uz to, git submoduli su također moćan alat iz razloga što jedan naš golem projekt smo u mogućnosti rastaviti u više manjih sekcija što doprinosti preglednosti i lakšem radu suradnika pri izradnji projekta.
\newpage
\item{\textbf{Popis Literature}}

//ovdje pisi literaturu i imas ovaj myref.bib file za to



\end{enumerate}
\end{document}
